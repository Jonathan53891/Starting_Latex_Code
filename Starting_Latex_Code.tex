\documentclass[12pt,a4paper]{extarticle}
\usepackage[a4paper,margin=6mm]{geometry}
\usepackage{amsmath}
\usepackage{hyperref}

\title{Homework 2}
\author{Jonathan Pilafas}

\begin{document}

\maketitle

\section*{17)} 

a) We know that if A is a square matrix order \textit{n}, then the determinant of A, or det(A), by \textit{i}th row expansion is:
\\
\\
det(A) = $a_{i1}$$C_{i1}$ + $a_{i2}$$C_{i2}$ + $a_{i3}$$C_{i3}$ + $a_{i4}$$C_{i4}$ + ... + $a_{in}$$C_{in}$, where $C_{ij}$ is the co-factor of the matrix A.
\\
\\
Given the matrix
\[
	A = \begin{bmatrix}
		-3 & 2 & 1 \\
		4 & 5 & 6 \\
		2 & -3 & 1
	\end{bmatrix}
\]
In question 15, we already found the co-factors, and we will use
the second row to find the determinant. So we have:
\begin{center}
$C_{21}$ = -5 \\
$C_{22}$ = -5 \\
$C_{23}$ = -5
\end{center}
Thus, we can find the determinant:
\begin{center}
det(A) = $a_{21}$$C_{21}$ + $a_{22}$$C_{22}$ + $a_{23}$$C_{23}$\\
det(A) = (4)(-5)+(5)(-5)+(6)(-5) \\
det(A) = -20 - 25 - 30
\[
\boxed{det(A) = -75}
\]
\end{center}
\\
b) We need to find the determinant by the second column. If A is a square matrix of order n, then det(A) with respect to the expansion by jth column is given by:
\\
\\
det(A) = $a_{1j}$$C_{1j}$ + $a_{2j}$$C_{2j}$ + $a_{3j}$$C_{3j}$ + $a_{4j}$$C_{4j}$ + ... + $a_{nj}$$C_{nj}$
\\
\\
We know from question 15 that
\begin{center}
$C_{12}$ = 8 \\
$C_{22}$ = -5 \\
$C_{32}$ = 22
\end{center}
Thus, we can find the determinant:
\begin{center}
det(A) = $a_{12}$$C_{12}$ + $a_{22}$$C_{22}$ + $a_{32}$$C_{22}$\\
det(A) = (2)(8)+(5)(-5)+(-3)(22) \\
det(A) = 16 - 25 - 66
\[
\boxed{det(A) = -75}
\]
\end{center}
\newpage

\section*{18)} 
To begin, we must find the co-factors of the given matrix in question 16.
\\
\\
If is A is a square matrix, then the minor $M_{ij}$ of the entry $a_{ij}$ is the determinant of the matrix obtained by deleting the \textit{i}th row and \textit{j}th column of A. And the co-factor $C_{ij}$ of the entry $a_{ij}$ is $C_{ij}$ = $(-i)^{i+j}$$M_{ij}$.
\\
\\
Given the matrix
\[
	A = \begin{bmatrix}
		-3 & 4 & 2 \\
		6 & 3 & 1 \\
		4 & -7 & -8
	\end{bmatrix}
\]
\begin{center} 
AND 
\\
det(A) = ad - bc (for 2x2 matrices)
\end{center}

$M_{11}$ = det([\begin{bmatrix}
3&1\\
-7&-8\\
\end{bmatrix}]) = ad - bc = (3)(-8) - (1)(-7) = -24 - (-7) = -17

$M_{12}$ = det([\begin{bmatrix}
6&1\\
4&-8\\
\end{bmatrix}]) = ad - bc = (6)(-8) - (1)(4) = -48 - 4 = -52

$M_{13}$ = det([\begin{bmatrix}
3&1\\
-7&-8\\
\end{bmatrix}]) = ad - bc = (3)(-8) - (1)(-7) = -24 - (-7) = -17

$M_{11}$ = det([\begin{bmatrix}
3&1\\
-7&-8\\
\end{bmatrix}]) = ad - bc = (3)(-8) - (1)(-7) = -24 - (-7) = -17

$M_{11}$ = det([\begin{bmatrix}
3&1\\
-7&-8\\
\end{bmatrix}]) = ad - bc = (3)(-8) - (1)(-7) = -24 - (-7) = -17

$M_{11}$ = det([\begin{bmatrix}
3&1\\
-7&-8\\
\end{bmatrix}]) = ad - bc = (3)(-8) - (1)(-7) = -24 - (-7) = -17

$M_{11}$ = det([\begin{bmatrix}
3&1\\
-7&-8\\
\end{bmatrix}]) = ad - bc = (3)(-8) - (1)(-7) = -24 - (-7) = -17

$M_{11}$ = det([\begin{bmatrix}
3&1\\
-7&-8\\
\end{bmatrix}]) = ad - bc = (3)(-8) - (1)(-7) = -24 - (-7) = -17

$M_{11}$ = det([\begin{bmatrix}
3&1\\
-7&-8\\
\end{bmatrix}]) = ad - bc = (3)(-8) - (1)(-7) = -24 - (-7) = -17

a) We know that if A is a square matrix order \textit{n}, then the determinant of A, or det(A), by \textit{i}th row expansion is:
\\
\\
det(A) = $a_{i1}$$C_{i1}$ + $a_{i2}$$C_{i2}$ + $a_{i3}$$C_{i3}$ + $a_{i4}$$C_{i4}$ + ... + $a_{in}$$C_{in}$, where $C_{ij}$ is the co-factor of the matrix A.
\\
\\
Given the matrix
\[
	A = \begin{bmatrix}
		-3 & 2 & 1 \\
		4 & 5 & 6 \\
		2 & -3 & 1
	\end{bmatrix}
\]
In question 15, we already found the co-factors, and we will use
the second row to find the determinant. So we have:
\begin{center}
$C_{21}$ = -5 \\
$C_{22}$ = -5 \\
$C_{23}$ = -5
\end{center}
Thus, we can find the determinant:
\begin{center}
det(A) = $a_{21}$$C_{21}$ + $a_{22}$$C_{22}$ + $a_{23}$$C_{23}$\\
det(A) = (4)(-5)+(5)(-5)+(6)(-5) \\
det(A) = -20 - 25 - 30
\[
\boxed{det(A) = -75}
\]
\end{center}
\\
b) We need to find the determinant by the second column. If A is a square matrix of order n, then det(A) with respect to the expansion by jth column is given by:
\\
\\
det(A) = $a_{1j}$$C_{1j}$ + $a_{2j}$$C_{2j}$ + $a_{3j}$$C_{3j}$ + $a_{4j}$$C_{4j}$ + ... + $a_{nj}$$C_{nj}$
\\
\\
We know from question 15 that
\begin{center}
$C_{12}$ = 8 \\
$C_{22}$ = -5 \\
$C_{32}$ = 22
\end{center}
Thus, we can find the determinant:
\begin{center}
det(A) = $a_{12}$$C_{12}$ + $a_{22}$$C_{22}$ + $a_{32}$$C_{22}$\\
det(A) = (2)(8)+(5)(-5)+(-3)(22) \\
det(A) = 16 - 25 - 66
\[
\boxed{det(A) = -75}
\]
\end{center}
\newpage






\section{Equation arrays}

In the flow of a fluid film we may report
\begin{eqnarray}
	u_\alpha & = & \epsilon^2 \kappa_{xxx} 
	\left( y-\frac{1}{2}y^2 \right),
	\label{equ}  \\
	v & = & \epsilon^3 \kappa_{xxx} y\,,
	\label{eqv}  \\
	p & = & \epsilon \kappa_{xx}\,.
	\label{eqp}
\end{eqnarray}
Alternatively, the curl of a vector field $(u,v,w)$ may be written 
with only one equation number:
\begin{eqnarray}
	\omega_1 & = &
	\frac{\partial w}{\partial y}-\frac{\partial v}{\partial z}\,,
	\nonumber  \\
	\omega_2 & = & 
	\frac{\partial u}{\partial z}-\frac{\partial w}{\partial x}\,,
	\label{eqcurl}  \\
	\omega_3 & = & 
	\frac{\partial v}{\partial x}-\frac{\partial u}{\partial y}\,.
	\nonumber
\end{eqnarray}
Whereas a derivation may look like
\begin{eqnarray*}
	(p\wedge q)\vee(p\wedge\neg q) & = & p\wedge(q\vee\neg q)
	\quad\text{by distributive law}  \\
	 & = & p\wedge T \quad\text{by excluded middle}  \\
	 & = & p \quad\text{by identity}
\end{eqnarray*}






\section{Functions}

Observe that trigonometric and other elementary functions are typeset 
properly, even to the extent of providing a thin space if followed by 
a single letter argument:
\[
	\exp(i\theta)=\cos\theta +i\sin\theta\,,\quad
	\sinh(\log x)=\frac{1}{2}\left( x-\frac{1}{x} \right).
\]
With sub- and super-scripts placed properly on more complicated 
functions,
\[
	\lim_{q\to\infty}\|f(x)\|_q 
	=\max_{x}|f(x)|,
\]
and large operators, such as integrals and
\begin{eqnarray*}
	e^x & = & \sum_{n=0}^\infty \frac{x^n}{n!}
	\quad\text{where }n!=\prod_{i=1}^n i\,,  \\
	\overline{U_\alpha} & = & \bigcap_\alpha U_\alpha\,.
\end{eqnarray*}
In inline mathematics the scripts are correctly placed to the side in 
order to conserve vertical space, as in
\(
	1/(1-x)=\sum_{n=0}^\infty x^n.
\)






\section{Accents}

Mathematical accents are performed by a short command with one 
argument, such as
\[
	\tilde f(\omega)=\frac{1}{2\pi}
	\int_{-\infty}^\infty f(x)e^{-i\omega x}\,dx\,,
\]
or
\[
	\dot{\vec \omega}=\vec r\times\vec I\,.
\]





\section{Command definition}

\newcommand{\Ai}{\operatorname{Ai}} 
The Airy function, $\Ai(x)$, may be incorrectly defined as this 
integral
\[
	\Ai(x)=\int\exp(s^3+isx)\,ds\,.
\]

\newcommand{\D}[2]{\frac{\partial #2}{\partial #1}}
\newcommand{\DD}[2]{\frac{\partial^2 #2}{\partial #1^2}}
\renewcommand{\vec}[1]{\boldsymbol{#1}}

This vector identity serves nicely to illustrate two of the new 
commands:
\[
	\vec\nabla\times\vec q
	=\vec i\left(\D yw-\D zv\right)
	+\vec j\left(\D zu-\D xw\right)
	+\vec k\left(\D xv-\D yu\right).
\]

Recall that typesetting multi-line mathematics is an art normally too hard for computer recipes.  Nonetheless, if you need to be automatically flexible about multi-line mathematics, and you do not mind some rough typesetting, then perhaps invoke \verb|\parbox| to help as follows: 
% The \verb|breqn| package is not yet reliable enough for general use.
\newcommand{\parmath}[2][0.8\linewidth]{\parbox[t]{#1}%
    {\raggedright\linespread{1.2}\selectfont\(#2\)}}
\[
u_1=\parmath{ -2 \gamma  \epsilon^{2} s_{2}+\mu  \epsilon^{3} \big( \frac{3}{8} s_{2}+\frac{1}{8} s_{1} i\big)+\epsilon^{3} \big( -\frac{81}{32} s_{4} s_{2}^{2}-\frac{27}{16} s_{4} s_{2} s_{1} i+\frac{9}{32} s_{4} s_{1}^{2}+\frac{27}{32} s_{3} s_{2}^{2} i-\frac{9}{16} s_{3} s_{2} s_{1}-\frac{3}{32} s_{3} s_{1}^{2} i\big) +\int_a^b 1-2x+3x^2-4x^3\,dx }
\]
Also, sometimes use \verb|\parbox| to typeset multiline entries in tables.


\section{Theorems et al.}

\newtheorem{theorem}{Theorem}
\newtheorem{corollary}[theorem]{Corollary}
\newtheorem{lemma}[theorem]{Lemma}
\newtheorem{definition}[theorem]{Definition}

\begin{definition}[right-angled triangles] \label{def:tri}
A \emph{right-angled triangle} is a triangle whose sides of length~\(a\), \(b\) and~\(c\), in some permutation of order, satisfies \(a^2+b^2=c^2\).
\end{definition}

\begin{lemma} 
The triangle with sides of length~\(3\), \(4\) and~\(5\) is right-angled.
\end{lemma}

This lemma follows from the Definition~\ref{def:tri} as \(3^2+4^2=9+16=25=5^2\).

\begin{theorem}[Pythagorean triplets] \label{thm:py}
Triangles with sides of length \(a=p^2-q^2\), \(b=2pq\) and \(c=p^2+q^2\) are right-angled triangles.
\end{theorem}

Prove this Theorem~\ref{thm:py} by the algebra \(a^2+b^2 =(p^2-q^2)^2+(2pq)^2
=p^4-2p^2q^2+q^4+4p^2q^2
=p^4+2p^2q^2+q^4
=(p^2+q^2)^2 =c^2\).


\end{document}
